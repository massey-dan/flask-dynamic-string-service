\documentclass[12pt]{article}
\usepackage[margin=1in]{geometry}
\usepackage{graphicx}
\usepackage{listings}
\usepackage{color}
\usepackage{hyperref}
\usepackage{titlesec}
\usepackage{fancyhdr}
\usepackage{parskip}

\pagestyle{fancy}
\fancyhf{}
\rhead{Cloud Dynamic Page}
\lhead{Python Flask + IaC}
\rfoot{\thepage}

\titleformat{\section}{\large\bfseries}{\thesection}{1em}{}
\titleformat{\subsection}{\normalsize\bfseries}{\thesubsection}{1em}{}

\definecolor{codegray}{gray}{0.9}
\lstset{
  backgroundcolor=\color{codegray},
  basicstyle=\ttfamily\footnotesize,
  frame=single,
  breaklines=true
}

\title{Dynamic HTML Page on the Cloud using Flask and Infrastructure as Code}
\author{Your Name}
\date{\today}

\begin{document}

\maketitle

\section{Overview}
This document outlines the architecture and implementation of a lightweight Flask application deployed on a cloud platform using Infrastructure as Code (IaC). The service dynamically displays a string on an HTML page and allows it to be updated without redeploying the application.

\section{Solution Summary}
\subsection*{Technology Stack}
\begin{itemize}
    \item \textbf{Backend:} Python Flask
    \item \textbf{Cloud Platform:} AWS / Azure / GCP (choose one)
    \item \textbf{IaC:} Terraform / AWS CloudFormation / Azure Bicep (choose one)
    \item \textbf{Hosting:} EC2 / App Service / GCP Compute Engine
\end{itemize}

\subsection*{How It Works}
\begin{enumerate}
    \item Flask app serves a dynamic HTML page with content from a file or memory.
    \item A REST endpoint \texttt{/update} accepts JSON payloads to update the string.
    \item The app is deployed using IaC to provision resources reproducibly.
\end{enumerate}

\section{Code Snippets}
\subsection*{Flask Route}
\begin{lstlisting}[language=Python]
@app.route("/")
def home():
    with open("message.txt", "r") as f:
        return f"<h1>The saved string is {f.read().strip()}</h1>"

@app.route("/update", methods=["POST"])
def update():
    data = request.get_json()
    new_string = data.get("dynamic_string")
    with open("message.txt", "w") as f:
        f.write(new_string)
    return jsonify({"message": "String updated"})
\end{lstlisting}

\subsection*{Terraform Example (AWS EC2)}
\begin{lstlisting}
resource "aws_instance" "flask_app" {
  ami           = "ami-xxxxxx"
  instance_type = "t2.micro"
  ...
  user_data     = <<-EOF
    #!/bin/bash
    sudo apt update
    sudo apt install -y python3-pip
    pip3 install flask
    python3 /home/ubuntu/app.py &
  EOF
}
\end{lstlisting}

\section{Design Decisions}
\subsection*{Why Flask?}
Flask offers a simple, lightweight framework ideal for rapid prototyping and small services.

\subsection*{Why IaC?}
Infrastructure as Code ensures repeatability, versioning, and easier management of cloud resources.

\subsection*{String Storage}
The string is stored in a plain text file on disk. This method is simple, persistent across requests, and sufficient for single-instance deployments.

\section{Trade-offs and Alternatives}
\begin{itemize}
    \item \textbf{Alternative Storage:} Redis, DynamoDB, or environment variables.
    \item \textbf{Scaling:} Current setup is single-instance. For horizontal scaling, centralized storage or a distributed cache is needed.
    \item \textbf{Security:} Currently, no authentication on \texttt{/update} route. Production systems should secure this endpoint.
\end{itemize}

\section{Future Improvements}
\begin{itemize}
    \item Add authentication to the update endpoint.
    \item Store the dynamic string in a database.
    \item Deploy behind a load balancer with autoscaling.
    \item Use Docker for containerized deployments.
\end{itemize}

\section{Conclusion}
This project demonstrates a basic dynamic web service using Python Flask, deployed on the cloud with Infrastructure as Code. It balances simplicity and functionality, and provides a foundation for building scalable, cloud-native applications.

\end{document}