\documentclass[12pt]{article}
\usepackage[margin=1in]{geometry}
\usepackage{graphicx}
\usepackage{listings}
\usepackage{color}
\usepackage{hyperref}
\usepackage{titlesec}
\usepackage{parskip}


\titleformat{\section}{\large\bfseries}{\thesection}{1em}{}
\titleformat{\subsection}{\normalsize\bfseries}{\thesubsection}{1em}{}

\definecolor{codegray}{gray}{0.9}
\lstset{
  backgroundcolor=\color{codegray},
  basicstyle=\ttfamily\footnotesize,
  frame=single,
  breaklines=true
}

\title{Dynamic String in a HTML page}
\author{Dan Massey}

\begin{document}

\maketitle

\section{Overview}
I built a Flask application and deployed it to AWS EC2 with Terraform. 
The application serves a HTML document with the string, and has an endpoint open to update it. 

\section{Solution Summary}
\subsection*{Technology Stack}
\begin{itemize}
    \item \textbf{Backend:} Python Flask
    \item \textbf{Cloud Platform:} AWS (EC2)
    \item \textbf{IaC:} Terraform
\end{itemize}

\subsection*{How It Works}
\begin{enumerate}
    \item Flask app serves a dynamic HTML page with a string loaded from memory
    \item A REST endpoint \texttt{/update} accepts JSON payloads to update the string.
\end{enumerate}

\section{Design Decisions}
\subsection*{Why Flask?}
Flask is a very simple and lightweight framework which made it ideal for a small service like this. 

\subsection*{Why Terraform/AWS}
There would have been more lightweight and easier solutions but I chose Terraform and AWS as they were mentioned in the job specification, and I have some experience in both.

\section{Useage}
Open the page at \texttt{http://18.170.218.128:5000/} to view the string. 

To update the string, send a \texttt{POST} request to \texttt{http://18.170.218.128:5000/update} with the body:
\begin{lstlisting}
{
    "dynamic_string": "I've updated the string!"
}
\end{lstlisting}

Note: I may have stopped the EC2 instance.

\section{Trade-offs and Alternatives}
\begin{itemize}
    \item \textbf{Alternative Storage:} It would have been possible to persist the storage of the string to a DB or file, but this would have meant added complexity and was beyond the scope of the task. However this could have allowed for better scalling as it would have made it possible to be multi-instanced.
    \item \textbf{Security:} Currently, there is no authentication on \texttt{/update} route. It would be better to secure this and have some form of validation on what the string was being set to. 
\end{itemize}

\section{Future Improvements}
\begin{itemize}
    \item Add authentication to the update endpoint.
    \item Store the dynamic string in a database.
    \item Add some validation on the string. 
\end{itemize}

\end{document}